\documentclass[11pt, oneside, parskip=half-]{scrbook}

\usepackage[linear, secthm]{handout}
\usepackage{scrhack}
\usepackage[normalem]{ulem}


%% New Commands %%
\newcommand{\lecture}[1]{\noindent\rule[0.35em]{0.4\textwidth}{1pt} \hfill \fbox{\large\textsc{Lecture #1}} \hfill \rule[0.35em]{0.4\textwidth}{1pt}\addcontentsline{toc}{chapter}{\protect\numberline{}Lecture #1}}

% Ch01 - Axiomatic Set Theory %
\DeclareMathOperator{\id}{id}
\DeclareMathOperator{\lxor}{\veebar}
\DeclareMathOperator{\lnand}{\uparrow}
\usepackage{stmaryrd}
\DeclareMathOperator{\contra}{\lightning}
% \DeclareMathOperator{\im}{im} % already defined as \img
\newcommand{\0}{\emptyset}
\DeclareMathOperator{\preimg}{preim}
\DeclareMathAlphabet{\euscr}{U}{eus}{m}{n}
\DeclareMathOperator{\powerset}{\euscr{P}}
\DeclareMathOperator{\setIso}{\cong_{\text{set}}}
\usetikzlibrary{cd}
\renewcommand{\nsim}{\not\sim}
\newcommand{\into}{\hookrightarrow}

% Ch02 - Topological Spaces %
\renewcommand{\O}{\mathcal{O}}
\newcommand{\Ostd}{\mathcal{O}_{\text{std.}}}
\newcommand{\Oq}[1]{\mathcal{O}_{\faktor{#1}{\sim}}}
\DeclareMathOperator{\topIso}{\cong_{\text{top.}}}
\DeclareMathOperator{\supp}{supp}
\DeclareMathOperator{\dcup}{\dot{\cup}}
\DeclareMathOperator{\grpIso}{\cong_{\text{grp}}}

% Ch03 - Topological Manifolds and Bundles %
\usetikzlibrary{decorations.markings}
\DeclareMathOperator{\bdlIso}{\cong_{\text{bdl}}}
\DeclareMathOperator{\proj}{proj}
\newcommand{\A}{\mathscr{A}}
\newcommand{\I}{\mathrm{i}}

% Ch04 - Differentiable Manifolds %
\usepackage{bbding}
\newcommand{\flower}{{\scalebox{0.75}{\FiveFlowerOpen} }}
\DeclareMathOperator{\diffIso}{\cong_{\text{diff}}}
\DeclareMathOperator{\vecIso}{\cong_{\text{vec}}}
\DeclareMathOperator{\Hom}{Hom}
\DeclareSymbolFont{MnSyC}{U}{MnSymbolC}{m}{n}
\DeclareMathSymbol{\diamondplus}{\mathbin}{MnSyC}{"7C}
\DeclareMathSymbol{\diamonddot}{\mathbin}{MnSyC}{"7E}
\DeclareMathOperator{\End}{End}
\DeclareMathOperator{\Aut}{Aut}
\usepackage{tensor}
\DeclareMathOperator{\hateq}{\hat{=}}
\DeclareMathOperator{\sgn}{sgn}
\DeclareMathOperator{\vol}{vol}

\DeclareFontFamily{U}{MnSymbolC}{}
\DeclareFontShape{U}{MnSymbolC}{m}{n}{
    <-6>  MnSymbolC5
   <6-7>  MnSymbolC6
   <7-8>  MnSymbolC7
   <8-9>  MnSymbolC8
   <9-10> MnSymbolC9
  <10-12> MnSymbolC10
  <12->   MnSymbolC12}{}

\makeatletter
\def\input@path{{Ch01 - Axiomatic Set Theory/}{Ch02 - Topological Spaces/}{Ch03 - Topological Manifolds and Bundles/}{Ch04 - Differentiable Manifolds/}}
\makeatother


\title{
    \Huge Lecture Notes
}

\subtitle{
    \huge Geometrical Anatomy of Theoretical Physics \\[10pt]
    \LARGE\normalfont\sffamily Dr. Fredric P Schuller\thanks{YouTube \href{https://youtube.com/playlist?list=PLPH7f_7ZlzxTi6kS4vCmv4ZKm9u8g5yic&si=Uy5ciENkuiTlvx6X}{playlist}}
}

\author{
\Large Piyush Kumar Singh \thanks{{\href{https://iampiyushkrsingh.github.io}{iamPiyushKrSingh.github.io}}}
}

\date{
    \large December 2024
}

%% Image Path %%
\graphicspath{{images/}}
\svgpath{{images/}}

%% TOC Setup %%
\setcounter{tocdepth}{3}
\setcounter{secnumdepth}{3}

\changemaincolor{Emerald}
\changesecondcolor{Periwinkle}

\hypersetup{
    pdftitle={Geometrical Anatomy of Theoretical Physics},
	pdfsubject={Physics, Quantum Mechanics, Differential Geometry, Topology},
    pdfauthor={Piyush Kumar Singh, Fredric Schuller},
    pdfkeywords={Quantum Mechanics, physics, qm, topology, geometry, notes, differentialbe manifold, theoretical physics, geometric anatomy},
    linktocpage=true
}

\begin{document}
\frontmatter
\begin{titlepage}
    \let\newpage\relax
    \singhtitle
\end{titlepage}

\chapter{Structure of this Course}
\noindent In theoretical physics we mainly deal with three big part of mathematics, namely, `Analysis', `Algebra' and `Geometry'. And at the mutual intersection of these three, we have different branches of physics like `Quantum Mechanics', `General Relativity', `Statistical Mechanics', etc. see \cref{fig:structure}.

\begin{figure}[H]
    \centering
    \includegraphics[width=0.4\textwidth]{lec01-math_struct.png}
    \caption{Structure of Theoretical Physics}
    \label{fig:structure}
\end{figure}\noindent
This course mainly focuses on differential geometry and topology, and their applications in theoretical physics. And we will start with the basic propositional logic and set theory, and then move to the topology and geometry of manifolds, and then we will see the applications of these in physics. Following is the structure of this course:

\begin{itemize}
    \item Logic
    \item Set Theory
    \item Topology
    \item Topological Manifolds
    \item Differential Manifolds
    \item Bundles
    \item Geometry: Symplectic Geometry, Metric Geometry, etc.
    \item Physics: Classical Mechanics, Electrodynamics, Quantum Mechanics, Statistical Mechanics, Special and General Relativity, etc.
\end{itemize}


\tableofcontents
\addcontentsline{toc}{chapter}{Contents}

\mainmatter

% chapter 01 - Axiomatic Set Theory
\input{ch01.tex}

% chapter 02 - Topological Spaces
\input{ch02.tex}

% chapter 03 - Topological Manifolds and Bundles
\input{ch03.tex}

% chapter 04 - Differentiable Manifolds
\chapter{Differentiable Manifolds}

\lecture{7}

\section{Adding Structure by refining the (maximal) \texorpdfstring{\(\SC^0\)}{C0}-atlas}

\begin{definition}[\(\flower\)-Atlas]
	Let \((M, \O)\) be a \(d\)-dimensional manifold. An atlas \(\A\) is called \(\flower\)-atlas, if any two charts \((U, x), (V, y) \in \A\) are \(\flower\)-compatible.
\end{definition}

In other words, either \(U \cap V = \0\) or if \(U \cap V \neq \0\), then the transition map \(y \circ x^{-1}\) is \(\flower\) as a map from \(\R^d\) to \(\R^d\).
\begin{figure}[H]
	\centering
	\begin{tikzcd}
		& U \cap V \arrow[ld, "x"'] \arrow[rd, "y"] & \\
		\R^d \supseteq x(U \cap V) \arrow[rr, shift left, "y \circ x\inv"] & & y(U \cap V) \subseteq \R^d \arrow[ll, shift left, "x \circ y\inv"]
	\end{tikzcd}
\end{figure}

Now, we can define the placeholder symbol \(\flower\) as:
\begin{itemize}
	\item \(\flower = \SC^0\): see \cref{def:c0-atlas}.
	\item \(\flower = \SC^k\): the transition map is \(k\)-times continuously differentiable as maps \(\R^d \to \R^d\).
	\item \(\flower = \SC^{\infty}\): the transition map is smooth (infinitely many times differentiable); i.e., \(k\)-times continuously differentiable for all \(k \in \N\).
	\item \(\flower = \SC^{\omega}\): the transition map is real-analytic; \ie, it can be locally represented by a convergent power series.
	\item \(\flower = \SC^{\omega}_{\C}\): the transition map is complex-analytic; equivalently, it satisfies the Cauchy-Riemann conditions.
\end{itemize}

\noindent Here for completeness, we need to define what are the Cauchy-Riemann conditions:\\
Set theoretical we know that \(\C \setIso \R^2\). Let \(f: \C \to \C\) be a complex function defined as
\begin{equation}
	\begin{aligned}
		f: \C    & \to \C                       \\
		x + \I y & \mapsto u(x, y) + \I v(x, y)
	\end{aligned}
\end{equation}
where \(u, v: \R^2 \to \R\) are real-valued functions. Then the Cauchy-Riemann conditions says that \(f\) is complex-analytic at \(x_0 + \I y_0\) if and only if the following two conditions are satisfied:
\begin{enumerate}
	\item All the partial derivatives of \(u\) and \(v\) exist at \((x_0, y_0)\) and are continuous in a neighborhood of \((x_0, y_0)\).
	\item The following two equations are satisfied:
	      \begin{equation}
		      \pdv{u}{x} \qty(x_0, y_0) = \pdv{v}{y} \qty(x_0, y_0) \quad \text{and} \quad \pdv{u}{y} \qty(x_0, y_0) = -\pdv{v}{x} \qty(x_0, y_0).
	      \end{equation}
\end{enumerate}

\begin{theorem}[Whitney]
	Any maximal \(\SC^k\)-atlas (for any \(k \ge 1\)) contains a \(\SC^{\infty}\)-atlas. Moreover, any two maximal \(\SC^k\)-atlases that contains the same \(\SC^{\infty}\)-atlas are identical.
\end{theorem}
In other words, we refine (or remove) all the charts in a maximal \(\SC^k\)-atlas which are not \(\SC^{\infty}\)-compatible, and we get a maximal \(\SC^{\infty}\)-atlas. This is the reason why we can always work with \(\SC^{\infty}\)-atlases given that we are working with a differentiable manifold. Immediate consequence of this theorem is that if any result is true for a \(\SC^k\)-atlas for any \(k \ge 1\), then it is also true for a \(\SC^{\infty}\)-atlas.

But in the case of \(\SC^0\)-atlas, it may happen that it doesn't admit a \(\SC^1\)-atlas, and hence we cannot refine it to a \(\SC^{\infty}\)-atlas.

Hence, we will not make any distinction between \(\SC^k\)-manifolds and \(\SC^{\infty}\)-manifolds, and we will always work with \(\SC^{\infty}\)-manifolds.

\begin{definition}[\(\SC^{k}\)-manifold]
	A triple \((M, \O, \A)\) is called a \(\SC^{k}\)-manifold where
	\begin{itemize}
		\item \((M, \O)\) is a topological manifold.
		\item \(\A\) is a maximal \(\SC^{k}\)-atlas on \(M\).
	\end{itemize}
\end{definition}

\begin{definition}[Incompatible Atlases]
	Let two \(\flower\)-compactible atlases \(\A_1\) and \(\A_2\) on a topological manifold \((M, \O)\) be called compatible if \(\A_1 \cup \A_2\) is a $\flower$-atlas on \(M\). Otherwise, they are called incompatible.
\end{definition}

\begin{remark}
	A given topological manifold \((M, \O)\) can have different incompatible atlases.
\end{remark}

A simple example of incompatible atlases,
\begin{example}
	Let \(M = \R\) with the standard topology, and let \(\A_1 = \{(\R, \id_{\R})\}\) and \(\A_2 = \{(\R, a \xmapsto[]{x} \sqrt[3]{a})\}\). Since each atlas contains only one chart, they are trivially \(\SC^{\infty}\)-compatible as the transition map is the identity map in both cases. But \(\A_1 \cup \A_2\) is not a \(\SC^{\infty}\)-atlas on \(M\) because the transition maps \(\id_{\R} \circ x\inv \equiv a \mapsto a^3\) which is a smooth map, but the transition map \(x \circ \id_{\R}\inv \equiv a \mapsto \sqrt[3]{a}\) is not smooth as it is not differentiable at \(a = 0\). Hence, \(\A_1\) and \(\A_2\) are incompatible atlases on \(M\).
\end{example}

This example shows that we can equip the real line \(\R\) with different incompatible \(\SC^{\infty}\)-struictures. This sounds bad as we want to do physics on \(\R\), and we want to have a unique \(\SC^{\infty}\)-structure on it. But this is not a problem, as we are given an atlas by the definition of differentiable manifold.

\begin{definition}[Differentiability]
	Let \((M, \O_M, \A_M)\) and \((N, \O_N, \A_N)\) be two \(\SC^{k}\)-manifolds of dimension \(m\) and \(n\) respectively. A map \(f: M \to N\) is called \(\SC^{k}\)-differentiable at a point \(p \in M\) if there exists a chart \((U, x) \in \A_M\) around \(p\) and a chart \((V, y) \in \A_N\) around \(f(p)\) such that the map \((y \circ f \circ x^{-1}): x(U) \to y(V)\) is \(\SC^{k}\)-differentiable at \(x(p)\) as a map from \(\R^m\) to \(\R^n\).
	\begin{figure}[H]
		\centering
		\begin{tikzcd}
			M \supseteq U \arrow[rr, "f"] \arrow[dd, "x"'] & & N \subseteq V \arrow[dd, "y"] \\
			& & \\
			\R^m \supseteq x(U) \arrow[rr, "y \circ f \circ x^{-1}"] & & y(V) \subseteq \R^n
		\end{tikzcd}
	\end{figure}
	If \(f\) is \(\SC^{k}\)-differentiable at every point \(p \in M\), then we say that \(f\) is a \(\SC^{k}\)-differentiable map from \(M\) to \(N\).
\end{definition}

\begin{proposition}
	The definition of \(\SC^{k}\)-differentiability is independent of the choice of charts \((U, x) \in \A_M\) and \((V, y) \in \A_N\) \ie\ the definition is well-defined.
\end{proposition}
\begin{proof}
	Consider two charts \((U, x), (\tilde{U}, \tilde{x}) \in \A_M\) around \(p\) and two charts \((V, y), (\tilde{V}, \tilde{y}) \in \A_N\) around \(f(p)\). We need to show that if \(f\) is \(\SC^{k}\)-differentiable at \(p\) with respect to the charts \((U, x)\) and \((V, y)\), then it should also be \(\SC^{k}\)-differentiable with respect to the charts \((\tilde{U}, \tilde{x})\) and \((\tilde{V}, \tilde{y})\). Consider the following commutative diagram:
	\begin{figure}[H]
		\centering
		\begin{tikzcd}
			\R^m \supseteq \tilde{x}(U \cap \tilde{U}) \arrow[rrrr, "\tilde{y} \circ f \circ \tilde{x}^{-1}"] \arrow[dddd, bend right=75, "x \circ \tilde{x}^{-1}"'] & & & & \tilde{y}(V \cap \tilde{V}) \subseteq \R^n \\
			& & & & \\
			M \supseteq U \cap \tilde{U} \ni p \arrow[uu, "\tilde{x}"'] \arrow[rrrr, "f"] \arrow[dd, "x"] & & & & f(p) \in V \cap \tilde{V} \subseteq N \arrow[uu, "\tilde{y}"] \arrow[dd, "y"'] \\
			& & & & \\
			\R^m \supseteq x(U \cap \tilde{U}) \arrow[rrrr, "y \circ f \circ x^{-1}"] & & & & y(V \cap \tilde{V}) \subseteq \R^n \arrow[uuuu, bend right=75, "\tilde{y} \circ y\inv"']
		\end{tikzcd}
	\end{figure}
	We know that the transition maps \(x \circ \tilde{x}^{-1}\) and \(\tilde{y} \circ y^{-1}\) are \(\SC^{k}\)-differentiable as they are transition maps between charts in the same atlas. So the composition of the maps
	\begin{equation}
		\tilde{y} \circ f \circ \tilde{x}^{-1} = (\tilde{y} \circ y\inv) \circ (y \circ f \circ x\inv) \circ (x \circ \tilde{x}^{-1})
	\end{equation}
	is also \(\SC^{k}\)-differentiable as a composition of \(\SC^{k}\)-differentiable maps. Hence, \(f\) is \(\SC^{k}\)-differentiable at \(p\) with respect to the charts \((\tilde{U}, \tilde{x})\) and \((\tilde{V}, \tilde{y})\).
\end{proof}

\begin{definition}[Diffeomorphism]
	Let \((M, \O_M, \A_M)\) and \((N, \O_N, \A_N)\) be two \(\SC^{k}\)-manifolds. A map \(f: M \to N\) is called a \(\SC^{k}\)-diffeomorphism if it is a bijection and both \(f\) and its inverse \(f^{-1}: N \to M\) are \(\SC^{k}\)-differentiable.
\end{definition}

\begin{definition}[Diffeomorphic]
	Two \(\SC^{k}\)-manifolds \((M, \O_M, \A_M)\) and \((N, \O_N, \A_N)\) are called diffeomorphic if there exists a \(\SC^{k}\)-diffeomorphism \(f: M \to N\). Then we write
	\begin{equation}
		M \diffIso N \quad \text{or} \quad (M, \O_M, \A_M) ~\diffIso~ (N, \O_N, \A_N).
	\end{equation}
\end{definition}

With this new notation, we want to finally answer the question: whether, for instance
\begin{equation}
	(\R, \Ostd, \A_{1, \text{max}}) \diffIso (\R, \Ostd, \A_{2, \text{max}})
\end{equation}
where \(\A_{1, \text{max}}\) and \(\A_{2, \text{max}}\) are the maximal \(\SC^{\infty}\)-atlases on \(\R\) defined in the previous example.

In principle, we want to know, how many different differentiable structures are there on a given topological manifold \((M, \O)\) -- up to diffeomorphism? The answer to this question is not known in general, but we know that it depends on the dimension of the manifold \(M\).

\begin{theorem}[Radon-Moise]
	For \(d \le 3\), any two \(\SC^{\infty}\)-manifolds of dimension \(d\) are diffeomorphic if and only if they are homeomorphic. In other words, let \((M, \O_M, \A_M)\) and \((N, \O_N, \A_N)\) be two \(\SC^{\infty}\)-manifolds of dimension \(d\) and \(d \le 3\). Then
	\begin{equation}
		(M, \O_M, \A_M) \diffIso (N, \O_N, \A_N) \iff (M, \O_M) \topIso (N, \O_N).
	\end{equation}
	So in particular, if \((M, \O_M)\) and \((N, \O_N)\) are homeomorphic, then we have a unique \(\SC^{\infty}\)-structure on \(M\) and \(N\) up to diffeomorphism.
\end{theorem}

From the above theorem, we can say that given a topological manifold \((M, \O_M)\), there is a unique \(\SC^{\infty}\)-structure on it up to diffeomorphism if \(\dim M \le 3\).

The answer for \(d > 4\) (specifically for \(d \ne 4\)) is that there are finitely many different differentiable structures on a given \emph{compact} topological manifold \((M, \O_M)\) up to diffeomorphism. This answer is provided by \emph{surgery theory} (or obstruction theory). This is a collection of tools and techniques in topology with which one obtains a new manifold from given ones by performing surgery on them, \ie\ by cutting, replacing and gluing parts in such a way as to control topological invariants like the fundamental group. The idea is to understand all manifolds in dimensions higher than 4 by performing surgery systematically.

\begin{remark}[Good News for String Theorists]
	According to many string theorists, our space-time is a 10-dimensional manifold. Since we don't have a unique differentiable structure on a 10-dimensional manifold, so in principle, different differentiable structures can lead to different predictions in physics, which is not what we want. But the good news is that for \(d = 10\), there are only finitely many different differentiable structures, so we can decide which one is the correct for our space-time by performing finite number of experiments.
\end{remark}

For \(d = 4\), the situation is very different. In fact, the problem of classifying all smooth differentiable structures is still open. But we know following partial results:
\begin{itemize}
	\item Non-compact 4-manifolds:

	      There are uncountably many non-diffeomorphic \(\SC^{\infty}\)-structures, specifically on \(\R^4\).

	\item Compact 4-manifolds:

	      The classification is not yet complete, but one of the most interesting results is that there are countably many non-diffeomorphic \(\SC^{\infty}\)-structures on a given compact 4-manifold with \(b_2 > 18\) (where \(b_2\) is the second Betti number, which is a topological invariant of the manifold).
\end{itemize}
\begin{remark}[Betti Numbers]
	Betti numbers are topological invariants defined using homology groups (notion of algebraic topology). But, intiuitively, the \(k\)-th Betti number \(b_k\) of a topological space is the number of \(k\)-dimensional holes in it.
	\begin{itemize}
		\item \(b_0\) is the number of connected components;
		\item \(b_1\) is the number of 1-dimensional or ``circular'' holes;
		\item \(b_2\) is the number of 2-dimensional ``voids'' or ``cavities''.
		\item And so on.
	\end{itemize}
	For example, the 2-sphere \(S^2\) has \(b_0 = 1\), \(b_1 = 0\) and \(b_2 = 1\) as it has one connected component, no circular holes and one 2-dimensional cavity. And the 2-torus \(T^2\) has \(b_0 = 1\), \(b_1 = 2\) and \(b_2 = 1\) as it has one connected component, two circular holes (one equitorial and one meridional) and one 2-dimensional cavity.
\end{remark}

Key feature of a differentiable manifold is that there exists a ``\emph{tangent space}'' at each point of the manifold.
\\
\pagebreak
\lecture{8}

\section{Review of Vector Spaces}

\begin{definition}[(Algebraic) Field]
    Let \(K\) be a non-empty set with two binary operations \(+\) and \(\cdot\) (addition and multiplication) such that
    \begin{itemize}
        \item \((K, +)\) is an abelian group with identity element \(0\).
        \item \((K \setminus \{0\}, \cdot)\) is an abelian group with identity element \(1\).
        \item Multiplication is distributive over addition, \ie\ for all \(a, b, c \in K\),
              \begin{equation}
                  a \cdot (b + c) = a \cdot b + a \cdot c \quad \text{and} \quad (a + b) \cdot c = a \cdot c + b \cdot c.
              \end{equation}
    \end{itemize}
    Then \(K\) is called a field.
\end{definition}
Later we will use a set \(R\) equipped with two binary operations \(+\) and \(\cdot\) but fewer axioms than a field, and we will call it a \emph{ring}.
\begin{definition}[Ring]
    A ring is a set \(R\) equipped with two binary operations \(+\) and \(\cdot\) such that
    \begin{itemize}
        \item \((R, +)\) is an abelian group with identity element \(0\).
        \item \((R, \cdot)\) is a monoid with identity element \(1\).

              \ie\ multiplication is associative and has an identity element, but it need not be commutative.
        \item Multiplication is distributive over addition, \ie\ for all \(a, b, c \in R\),
              \begin{equation}
                  a \cdot (b + c) = a \cdot b + a \cdot c \quad \text{and} \quad (a + b) \cdot c = a \cdot c + b \cdot c.
              \end{equation}
    \end{itemize}
\end{definition}
Some examples which we are using from our school days:
\begin{itemize}
    \item \((\Z, +, \cdot)\) is a \emph{commutative} ring but not a field as it has no multiplicative inverses for all non-zero elements.
    \item \((\Q, +, \cdot)\) and \((\R, +, \cdot)\) are fields.
    \item \((\SM_{n}(\R), +, \circ)\) is a ring, where \(\SM_{n}(\R)\) is the set of \(n \times n\) real matrices with the usual matrix addition and multiplication. Matrix multiplication is associative but not commutative, and it has an identity element \(I_n\) (the identity matrix). But it does not have multiplicative inverses for all non-zero elements, so it is not a field.
\end{itemize}

\begin{definition}[Vector Space over a Field \(K\)]
    A vector space \(V\) over a field \((K, +, \cdot)\) is a set equipped with two operations \(\oplus: V \times V \to V\) (vector addition) and \(\odot: K \times V \to V\) (scalar multiplication) such that
    \begin{itemize}
        \item \((V, \oplus)\) is an abelian group with identity element \(\va{0}\) (the zero vector).
        \item The scalar multiplication satisfies the following properties
              \begin{enumerate}[(i)]
                  \item Let \(1\) be the multiplicative identity of the field \(K\). Then for all \(\va{v} \in V: 1 \odot \va{v} = \va{v}\).
                  \item \(\forall a, b \in K: \forall \va{v} \in V: a \odot (b \odot \va{v}) = (a \cdot b) \odot \va{v}\).
                  \item \(\forall a \in K: \forall \va{u}, \va{v} \in V: a \odot (\va{u} \oplus \va{v}) = a \odot \va{u} \oplus a \odot \va{v}\).
                  \item \(\forall a, b \in K: \forall \va{v} \in V: (a + b) \odot \va{v} = a \odot \va{v} \oplus b \odot \va{v}\).
              \end{enumerate}
    \end{itemize}
\end{definition}


\lecture{9}

\section{Tangent Spaces to a Manifold}

Let \(M\) be a smooth manifold (from now on, whenever we say a smooth manifold, the associated topology and atlas are always implied). Then we can construct the followwing vector space over \(\R\):
\begin{equation}
    \qty(\SC^\infty(M), +, \cdot)
\end{equation}
where \(\SC^\infty(M) := \qty{f : M \to \R \mid f \text{ is smooth}}\) is the set of all smooth real-valued functions on \(M\). The notion of smoothness is via smooth charts in the atlas of \(M\). The addition and scalar multiplication are defined pointwise, for all \(f, g \in \SC^\infty(M)\), \(\lambda \in \R\), and \(p \in M\), as
\begin{align}
    (f + g)(p)           & := f(p) + g(p)         \\
    (\lambda \cdot f)(p) & := \lambda \cdot f(p).
\end{align}
It is easy to check that \(\qty(\SC^\infty(M), +, \cdot)\) is indeed a vector space over \(\R\).

\begin{definition}[Directional Derivative]
    Let \(\gamma: \R \to M\) be a smooth curve\footnotemark through a point \(p \in M\), and WLOG let \(\gamma(0) = p\). Then the \emph{directional derivative operator} along \(\gamma\) at \(p\) is a map
    \footnotetext{Here the notion of smoothness is via charts in the atlas of \(M\), \ie, for all charts \((U, x)\) of \(M\) such that \(p \in U\), the composition \(x \circ \gamma: \R \to \R^d\) is a smooth map in the usual sense.}
    \begin{equation}
        X_{\gamma, p} : \SC^\infty(M) \to \R
    \end{equation}
    defined as
    \begin{equation}
        \SC^\infty(M) \ni f \mapsto X_{\gamma, p}(f) := (f \circ \gamma)'(0) \in \R.
    \end{equation}
\end{definition}
Note the composition \(f \circ \gamma\) is a map from \(\R\) to \(\R\), and hence the derivative is the usual derivative of real-valued functions of a real variable.

\begin{proposition}
    The directional derivative operator \(X_{\gamma, p}\) along a smooth curve \(\gamma\) through \(p\) is a linear map, \ie, for all \(f, g \in \SC^\infty(M)\) and \(\lambda, \mu \in \R\),
    \begin{equation}
        X_{\gamma, p}(\lambda f + \mu g) = \lambda X_{\gamma, p}(f) + \mu X_{\gamma, p}(g).
    \end{equation}
\end{proposition}
\begin{proof}
    This follows from the linearity of the usual derivative of real-valued functions of a real variable.
    \begin{align*}
        X_{\gamma, p}(\lambda f + \mu g) & = ((\lambda f + \mu g) \circ \gamma)'(0)                  \\
                                         & = (\lambda (f \circ \gamma) + \mu (g \circ \gamma))'(0)   \\
                                         & = \lambda (f \circ \gamma)'(0) + \mu (g \circ \gamma)'(0) \\
                                         & = \lambda X_{\gamma, p}(f) + \mu X_{\gamma, p}(g).
    \end{align*}
\end{proof}
In differential geometry, \(X_{\gamma, p}\) is usually called a \emph{tangent vector} to curve \(\gamma\) at point \(p\). Physically, we can think of \(X_{\gamma, p}\) as the velocity vector of a particle moving along the curve \(\gamma\) at point \(p\). To see this, let \(\gamma_1, \gamma_2: \R \to M\) be two smooth curves through \(p\) such that \(\gamma_1(0) = \gamma_2(0) = p\), and \(\gamma_1(t) = \gamma_2(2 t)\) for all \(t \in \R\). Let \(f \in \SC^\infty(M)\) be a smooth function. Then
\begin{equation}
    X_{\gamma_1, p}(f) = (f \circ \gamma_1)'(0) = (f \circ \gamma_2)'(0) \cdot 2 = 2 (f \circ \gamma_2)'(0) = 2 X_{\gamma_2, p}(f).
\end{equation}
This means that \(X_{\gamma_1, p} = 2 X_{\gamma_2, p}\). So, if we think of \(\gamma_1\) and \(\gamma_2\) as the trajectories of two particles moving through point \(p\) on manifold \(M\), then the velocity vector of the first particle at \(p\) is twice that of the second particle at \(p\), which is consistent with our physical intuition.

\begin{definition}[Tangent Vector Space]
    Let \(M\) be a smooth manifold and \(p \in M\). The \emph{tangent vector space} to \(M\) at \(p\), denoted by \(\ST_p M\), is defined as
    \begin{equation}
        \ST_p M := \qty{X_{\gamma, p} \mid \gamma: \R \to M \text{ is a smooth curve with } \gamma(0) = p}.
    \end{equation}
    Equipped with following operations:
    \begin{align*}
         & \oplus: \ST_p M \times \ST_p M \to \ST_p M, \\
         & \odot:  \R \times \ST_p M \to \ST_p M,
    \end{align*}
    defined pointwise as
    \begin{align*}
        (X_{\gamma_1, p} \oplus X_{\gamma_2, p})(f) & := X_{\gamma_1, p}(f) + X_{\gamma_2, p}(f), \quad \forall f \in \SC^\infty(M),                    \\
        (\lambda \odot X_{\gamma, p})(f)            & := \lambda \cdot X_{\gamma, p}(f), \quad \forall f \in \SC^\infty(M) \text{ and } \lambda \in \R,
    \end{align*}
    \(\qty(\ST_p M, \oplus, \odot)\) is a vector space over \(\R\).
\end{definition}
This definition is still incomplete, as the pointwise addition and scalar multiplication doesn't guarantee that the results are still in \(\ST_p M\). So, we have to prove the following proposition.
\begin{proposition}
    The operations \(\oplus\) and \(\odot\) defined above are closed in \(\ST_p M\), \ie, for all \(X_{\gamma_1, p}, X_{\gamma_2, p} \in \ST_p M\) and \(\lambda \in \R\),
    \begin{gather}
        X_{\gamma_1, p} \oplus X_{\gamma_2, p} \in \ST_p M, \\
        \lambda \odot X_{\gamma, p}            \in \ST_p M.
    \end{gather}
\end{proposition}
So we need to show that there exist smooth curves \(\gamma_3, \gamma_4: \R \to M\) such that \(\gamma_3(0) = \gamma_4(0) = p\) and
\begin{align*}
    X_{\gamma_3, p} & = X_{\gamma_1, p} \oplus X_{\gamma_2, p}, \\
    X_{\gamma_4, p} & = \lambda \odot X_{\gamma, p}.
\end{align*}
Since the notion of derivative is local, so if two curves agree on a neighborhood of \(0 \in \R\), then they have the same derivative at \(0\) \ie, if \(\gamma_1(t) = \gamma_2(t)\) for all \(t\) in some open interval containing \(0\), then \(X_{\gamma_1, p} = X_{\gamma_2, p}\). So, it is sufficient to construct \(\gamma_3\) and \(\gamma_4\) on some open interval containing \(0\).
\begin{proof}
    Let \((U, x)\) be a chart of \(M\) around \(p\), \ie, \(p \in U\) and \(x: U \to x(U) \subseteq \R^d\) is a homeomorphism.

    Let \(I \subseteq \R\) be an open interval containing \(0\) such that \(\gamma(t), \gamma_1(t), \gamma_2(t) \in U\) for all \(t \in I\). Such an interval exists since \(\gamma, \gamma_1, \gamma_2\) are continuous and \(\gamma(0) = \gamma_1(0) = \gamma_2(0) = p \in U\), and \(U\) is open in \(M\).
    \begin{enumerate}
        \item Construct a curve \(\gamma_3: I \to M\) using the chart \((U, x)\) as follows:
              \begin{equation}
                  \gamma_3(t) := x^{-1} \circ (x \circ \gamma_1(t) + x \circ \gamma_2(t) - x(p)), \quad \forall t \in I.
              \end{equation}
              Note that \(x \circ \gamma_1(t), x \circ \gamma_2(t) \in \R^d\) and \(x(p) \in \R^d\), so the addition and subtraction are well-defined. Also, since \(x\) is a diffeomorphism, \(x \circ \gamma_1\) and \(x \circ \gamma_2\) are smooth maps from \(I\) to \(\R^d\), and hence their sum is also a smooth map from \(I\) to \(\R^d\). Therefore, the composition \(\gamma_3\) is a smooth map from \(I\) to \(M\). Moreover, \(\gamma_3(0) = x^{-1}(x(p) + x(p) - x(p)) = p\).

              Now, for all \(f \in \SC^\infty(M)\),
              \begin{align*}
                  X_{\gamma_3, p}(f) & = (f \circ \gamma_3)'(0)                                                                                                                  \\
                                     & = \qty(f \circ x^{-1} \circ (x \circ \gamma_1 + x \circ \gamma_2 - x(p)))'(0)                                                             \\
                  \intertext{map \(f \circ x^{-1}: x(U) \subseteq \R^d \to \R\), and \(x \circ \gamma_1 + x \circ \gamma_2 - x(p): I \subseteq \R \to \R^d\), so we can apply the multivariate chain rule, taking derivative in \(j\)-th coordinate direction, \(j = 1, \ldots, d\):}
                                     & = \qty[\partial_j (f \circ x^{-1})(x(p))] \cdot \qty(x^j \circ \gamma_1 + x^j \circ \gamma_2 - x^j(p))'(0)                                \\
                  \shortintertext{where \(x^j\) is the \(j\)-th coordinate function of chart \(x\). Here sum over \(j\) from \(1\) to \(d\) is implied. Using linearity of the usual derivative, we have:}
                                     & = \qty[\partial_j (f \circ x^{-1})(x(p))] \cdot \qty((x^j \circ \gamma_1)'(0) + (x^j \circ \gamma_2)'(0))                                 \\
                                     & = [\partial_j (f \circ x^{-1})(x(p)) \cdot (x^j \circ \gamma_1)'(0)] + [\partial_j (f \circ x^{-1})(x(p)) \cdot (x^j \circ \gamma_2)'(0)] \\
                  \shortintertext{combining the term in the square brackets, we get:}
                                     & = (f \circ \gamma_1)'(0) + (f \circ \gamma_2)'(0)                                                                                         \\
                                     & = X_{\gamma_1, p}(f) + X_{\gamma_2, p}(f)                                                                                                 \\
                                     & = (X_{\gamma_1, p} \oplus X_{\gamma_2, p})(f).
              \end{align*}
              Since this is true for all \(f \in \SC^\infty(M)\), we have
              \begin{equation}
                  X_{\gamma_3, p} = X_{\gamma_1, p} \oplus X_{\gamma_2, p}.
              \end{equation}

        \item Construct a curve \(\gamma_4: I \to M\) using the chart \((U, x)\) as follows:
              \begin{equation}
                  \gamma_4(t) := x^{-1} \circ (x \circ \gamma(\lambda t)) , \quad \forall t \in I.
              \end{equation}
              Here it is tempting rewrite \(\gamma_4(t) = \gamma(\lambda t)\), but this lead to a problem that how to define \(f'(p)\) when we find \(X_{\gamma_4, p}(f)\).

              Note that since \(x\) is a diffeomorphism, \(x \circ \gamma\) is a smooth map from \(I\) to \(\R^d\), and hence the composition \(\gamma_4\) is also a smooth map from \(I\) to \(M\). Moreover, \(\gamma_4(0) = x^{-1}(x(p)) = p\).

              Now, for all \(f \in \SC^\infty(M)\),
              \begin{align*}
                  X_{\gamma_4, p}(f) & = (f \circ \gamma_4)'(0)                                                                   \\
                                     & = \qty(f \circ x^{-1} \circ (x \circ \gamma(\lambda t)))'(0)                               \\
                  \intertext{map \(f \circ x^{-1}: x(U) \subseteq \R^d \to \R\), and \(x \circ \gamma(\lambda t): I \subseteq \R \to \R^d\), so we can apply the multivariate chain rule, taking derivative in \(j\)-th coordinate direction, \(j = 1, \ldots, d\):}
                                     & = \qty[\partial_j (f \circ x^{-1})(x(p))] \cdot \qty(x^j \circ \gamma(\lambda t))'(0)      \\
                  \shortintertext{where \(x^j\) is the \(j\)-th coordinate function of chart \(x\). Using the chain rule for the usual derivative, we have:}
                                     & = \qty[\partial_j (f \circ x^{-1})(x(p))] \cdot \qty[(x^j \circ \gamma)'(0) \cdot \lambda] \\
                                     & = \lambda [\partial_j (f \circ x^{-1})(x(p)) \cdot (x^j \circ \gamma)'(0)]                 \\
                  \shortintertext{combining the term in the square brackets, we get:}
                                     & = \lambda (f \circ \gamma)'(0)                                                             \\
                                     & = \lambda X_{\gamma, p}(f)                                                                 \\
                                     & = (\lambda \odot X_{\gamma, p})(f).
              \end{align*}
              Since this is true for all \(f \in \SC^\infty(M)\), we have
              \begin{equation}
                  X_{\gamma_4, p} = \lambda \odot X_{\gamma, p}.
              \end{equation}
    \end{enumerate}
\end{proof}
\begin{remark}[Independence of Chart Choice]
    The construction of \(\gamma_3\) and \(\gamma_4\) depends on the choice of chart \((U, x)\). However, the resulting tangent vectors \(X_{\gamma_3, p}\) and \(X_{\gamma_4, p}\) do not depend on the choice of chart. This is because if we choose another chart \((V, y)\) around \(p\), then the transition map \(y \circ x^{-1}: x(U \cap V) \to y(U \cap V)\) is a diffeomorphism between open subsets of \(\R^d\), and hence the construction of \(\gamma_3\) and \(\gamma_4\) using chart \((V, y)\) will yield the same tangent vectors \(X_{\gamma_3, p}\) and \(X_{\gamma_4, p}\).
\end{remark}

\subsection{Algebras and Derivations}

\begin{definition}[Algebra over a Field]
    Let \((V, +, \cdot)\) be a vector space over a field \(K\) equipped with a ``product'' operation,
    \begin{equation}
        \bullet: V \times V \to V,
    \end{equation}
    such that \(\bullet\) is bilinear. Then \((V, +, \cdot, \bullet)\) is called an \emph{algebra over field \(K\)}.
\end{definition}
In the future, we will impose more conditions on the product operation \(\bullet\), such as anti-symmetry to get something called a Lie algebra. A typical example for that is the cross product in \(\R^3\).
\begin{example}[Algebra of Smooth Functions]
    We have already seen that \(\qty(\SC^\infty(M), +, \cdot)\) is a vector space over \(\R\). Now, we can define a product operation \(\bullet\) on \(\SC^\infty(M)\) as follows:
    \begin{equation}
        \begin{aligned}
            \bullet: \SC^\infty(M) \times \SC^\infty(M) & \to \SC^\infty(M),  \\
            (f, g)                                      & \mapsto f \bullet g
        \end{aligned}
    \end{equation}
    where \((f \bullet g)(p) := f(p) \cdot g(p)\) for all \(p \in M\). It is easy to check that \(\bullet\) is bilinear, and hence \(\qty(\SC^\infty(M), +, \cdot, \bullet)\) is an algebra over \(\R\).

    Note the difference between \(\cdot\) and \(\bullet\): the former is scalar multiplication, while the latter is function multiplication, both at heart uses the field multiplication in \(\R\).
\end{example}
Let's look at some special algebras where the product operation satisfies some special properties.
\begin{definition}
    Let \((V, +, \cdot, \bullet)\) be an algebra over a field \(K\). The algebra is called:
    \begin{itemize}
        \item \emph{Associative} if for all \(u, v, w \in V\),
              \begin{equation}
                  (u \bullet v) \bullet w = u \bullet (v \bullet w).
              \end{equation}
        \item \emph{Commutative} if for all \(u, v \in V\),
              \begin{equation}
                  u \bullet v = v \bullet u.
              \end{equation}
        \item \emph{Unital} if there exists an element \(\mathbf{1} \in V\) such that
              \begin{equation}
                  \mathbf{1} \bullet v = v \bullet \mathbf{1} = v, \quad \forall v \in V.
              \end{equation}
    \end{itemize}
\end{definition}
Now let's look at more important class of algebras, which are not necessarily associative or commutative.
\begin{definition}[Lie Algebra]
    A \emph{Lie algebra} over a field \(K\) is an algebra \((V, +, \cdot, \comm{\star}{\star})\) such that the product operation \(\comm{\star}{\star}\), called the \emph{Lie bracket}, satisfies the following properties:
    \begin{itemize}
        \item \emph{Antisymmetry}: for all \(u, v \in V\),
              \begin{equation}
                  \comm{u}{v} = -\comm{v}{u}.
              \end{equation}
        \item \emph{Jacobi Identity}: for all \(u, v, w \in V\),
              \begin{equation}
                  \comm{u}{\comm{v}{w}} + \comm{v}{\comm{w}{u}} + \comm{w}{\comm{u}{v}} = 0.
              \end{equation}
    \end{itemize}
    Note that the \(0\) here is the additive identity of the vector space \((V, +, \cdot)\).
\end{definition}
It is easy to see that for a non-trivial Lie bracket, the algebra cannot be unital.

\begin{definition}[Derivation]
    Let \((V, +, \cdot, \bullet)\) be an algebra over a field \(K\). A \emph{derivation} on \(V\) is a linear map \(D: V \to V\) such that it satisfies the Leibniz rule:
    \begin{equation}
        D(u \bullet v) = D(u) \bullet v + u \bullet D(v), \quad \forall u, v \in V.
    \end{equation}
\end{definition}
\begin{example}[Derivation on Smooth Functions]
    We have already seen that \((\SC^\infty(M), +, \cdot, \bullet)\) is an algebra over \(\R\). Fix a point \(p \in M\), take any tangent vector \(X_{\gamma, p} \in \ST_p M\). We know from the definition that \(X_{\gamma, p}: \SC^\infty(M) \to \R\) is a linear map. Now let's check if it satisfies the Leibniz rule, for all \(f, g \in \SC^\infty(M)\),
    \begin{align*}
        X_{\gamma, p}(f \bullet g) & = ((f \bullet g) \circ \gamma)'(0)                                                                \\
                                   & = ((f \circ \gamma) \cdot (g \circ \gamma))'(0)                                                   \\
                                   & = (f \circ \gamma)'(0) \cdot (g \circ \gamma)(0) + (f \circ \gamma)(0) \cdot (g \circ \gamma)'(0) \\
                                   & = X_{\gamma, p}(f) \cdot g(p) + f(p) \cdot X_{\gamma, p}(g).
    \end{align*}
    So, \(X_{\gamma, p}\) satisfies the Leibniz rule. However, note that \(X_{\gamma, p}(f)\) is a real number, not a smooth function on \(M\). So, \(X_{\gamma, p}\) is not a derivation on the algebra \((\SC^\infty(M), +, \cdot, \bullet)\), usually called a derivation at point \(p\).

    Now define a map
    \begin{equation}
        \begin{aligned}
            D: \SC^\infty(M) & \to \SC^\infty(M), \\
            f                & \mapsto D(f)
        \end{aligned}
    \end{equation}
    where \(D(f)(p) := X_{\gamma, p}(f)\) for all \(p \in M\). Then \(D\) is a derivation on the algebra \((\SC^\infty(M), +, \cdot, \bullet)\).
\end{example}

\begin{example}
    Let \(V\) be the vector space over \(\R\), define \(A := \End(V)\), we know that \((A, +, \cdot)\) is a vector space over \(\R\). Now define a product operation on \(A\) as follows:
    \begin{equation}
        \begin{aligned}
            \comm{\star}{\star}: A \times A & \to A,                                                          \\
            (\phi, \psi)                    & \mapsto \comm{\phi}{\psi} := \phi \circ \psi - \psi \circ \phi,
        \end{aligned}
    \end{equation}
    where \(\circ\) is the composition of linear maps. It is easy to see that \(\comm{\star}{\star}\) is bilinear, and hence \((A, +, \cdot, \comm{\star}{\star})\) is an algebra over \(\R\).
    Moreover, \(\comm{\star}{\star}\) is antisymmetric, and for all \(\phi, \psi, \rho \in A\),
    \begin{equation}
        \comm{\phi}{\comm{\psi}{\rho}} + \comm{\psi}{\comm{\rho}{\phi}} + \comm{\rho}{\comm{\phi}{\psi}} = 0,
    \end{equation}
    which is called the \emph{Jacobi identity}. So, \((A, +, \cdot, \comm{\star}{\star})\) is a Lie algebra over \(\R\).

    Now fix \(H \in A\), define a map
    \begin{equation}
        \begin{aligned}
            D_H : A & \to A,                               \\
            \phi    & \mapsto D_H(\phi) := \comm{H}{\phi}.
        \end{aligned}
    \end{equation}
    Let's check if \(D_H\) is a derivation, for all \(\phi, \psi \in A\),
    \begin{align*}
        D_H(\comm{\phi}{\psi}) & = \comm{H}{\comm{\phi}{\psi}}                                \\
        \shortintertext{using the Jacobi identity, we have:}
                               & = -\comm{\phi}{\comm{\psi}{H}} - \comm{\psi}{\comm{H}{\phi}} \\
        \shortintertext{rearranging the terms and use antisymmetry, we get:}
                               & = \comm{\comm{H}{\phi}}{\psi} + \comm{\phi}{\comm{H}{\psi}}  \\
                               & = \comm{D_H(\phi)}{\psi} + \comm{\phi}{D_H(\psi)}.
    \end{align*}
    So, \(D_H\) is a derivation on the Lie algebra \((A, +, \cdot, \comm{\star}{\star})\).
\end{example}
With this example, we can see the algebraic structure of Poisson brackets in classical mechanics, and the commutator in quantum mechanics.
\begin{remark}[Poisson Bracket]
    In classical mechanics, the state of a system is represented by a point in phase space (which is a symplectic manifold), and observables are represented by smooth functions on the phase space. The Poisson bracket defines a Lie algebra structure on the space of observables. If we fix an observable \(H\) (the Hamiltonian), then the map \(D_H(f) := \acomm{H}{f}\) is a derivation on the Lie algebra of observables, which generates the time evolution of the system according to Hamilton's equations.
\end{remark}
\begin{remark}[Commutator in Quantum Mechanics]
    Similarly, in quantum mechanics, the state of a system is represented by a vector in a Hilbert space, and observables are represented by self-adjoint operators on that space. The commutator defines a Lie algebra structure on the space of observables. If we fix an observable \(H\) (the Hamiltonian operator), then the map \(D_H(\phi) := \comm{H}{\phi}\) is a derivation on the Lie algebra of observables, which generates the time evolution of the system according to the Heisenberg equation of motion.
\end{remark}

\subsection{Basis and Dimension of Tangent Space}

We have shown that for a smooth manifold \(M\) and a point \(p \in M\), the set of tangent vectors \(\ST_p M\) is a vector space over \(\R\). Now we will prove a very crucial theorem in differential geometry, which states that the dimension of the tangent space \(\ST_p M\) is equal to the dimension of the manifold \(M\).

\begin{theorem}[Dimension of Tangent Space]
    Let \(M\) be a smooth manifold of dimension \(d\), then for all \(p \in M\), the tangent vector space \(\ST_p M\) is a vector space over \(\R\) of dimension \(d\).
    \begin{equation}
        \dim(\ST_p M) = \dim(M) = d.
    \end{equation}
\end{theorem}
Note that we have used the same symbol \(\dim\) for the dimension of a manifold and the dimension of a vector space, but they are different concepts. The dimension of a manifold is defined as the dimension of the Euclidean space that it locally resembles, while the dimension of a vector space is defined as the cardinality of its basis.
\begin{proof}
    Fix a point \(p \in M\), and fix a chart \((U, x)\) of \(M\) around \(p\).

    To prove this theorem, we will construct a basis of \(\ST_p M\) consisting of \(d\) tangent vectors.

    Define \(d\) curves \(\gamma_j: \R \to U\), \(j = 1, \ldots, d\), such that
    \begin{equation}
        \gamma_j(0) = p; \qquad \qquad \qquad \qquad x^i \circ \gamma_j(t) = \delta_j^i t, \quad \forall t \in \R,
    \end{equation}
    where \(x^i\) is the \(i\)-th coordinate function of chart \(x\), and \(\delta_j^i\) is the Kronecker delta. So pictorially, \(\gamma_j\) is a curve that moves along the \(j\)-th coordinate axis in the Euclidean space \(\R^d\) under the chart \(x\).

    Name the corresponding tangent vectors at \(p\) as
    \begin{equation}
        e_j := X_{\gamma_j, p}, \quad j = 1, \ldots, d.
    \end{equation}
    Let's look at how \(e_j\) acts on a smooth function \(f \in \SC^\infty(M)\):
    \begin{align*}
        e_j(f) & = (f \circ \gamma_j)'(0) = \qty(f \circ \id_U \circ \gamma_j)'(0)            \\
        \shortintertext{insert the identity map \(\id_U = x^{-1} \circ x\) on \(U\):}
               & = \qty(f \circ x^{-1} \circ (x \circ \gamma_j))'(0)                          \\
               & = \qty[\partial_i (f \circ x^{-1})(x(p))] \cdot \qty(x^i \circ \gamma_j)'(0) \\
               & = \qty[\partial_i (f \circ x^{-1})(x(p))] \cdot \delta_j^i                   \\
               & = \partial_j (f \circ x^{-1})(x(p)).
    \end{align*}
    Define a formal symbol as
    \begin{equation}
        \qty(\pdv{x^j})_p (f) := \partial_j (f \circ x^{-1})(x(p)), \quad \forall f \in \SC^\infty(M).
    \end{equation}
    Don't confuse this notation with the usual partial derivative of a function of several real variables.
    \begin{align*}
         & \qty(\partial_j)_p: \SC^\infty(\R^d, \R) \to \R, \\
         & \qty(\pdv{x^j})_p: \SC^\infty(M) \to \R.
    \end{align*}
    So we have
    \begin{equation}
        e_j = \qty(\pdv{x^j})_p, \quad j = 1, \ldots, d.
    \end{equation}
    Define the set
    \begin{equation}
        \SB := \qty{e_1, \ldots, e_d} = \qty{\qty(\pdv{x^1})_p, \ldots, \qty(\pdv{x^d})_p}.
    \end{equation}
    We will show that \(\SB\) is a basis of \(\ST_p M\), \ie, for any \(X \in \ST_p M\), there exist unique real numbers \(X^1, \ldots, X^d\) such that
    \begin{equation}
        X = X^j e_j = X^j \qty(\pdv{x^j})_p. \qquad \text{(sum over \(j\) from \(1\) to \(d\) is implied)}
    \end{equation}
    \begin{enumerate}
        \item \textbf{Spanning}: We know \(\exists \gamma: \R \to M\) such that \(X = X_{\gamma, p}\). For all \(f \in \SC^\infty(M)\),
              \begin{align*}
                  X(f) & = (f \circ \gamma)'(0) = \qty(f \circ \id_U \circ \gamma)'(0)              \\
                  \shortintertext{insert the identity map \(\id_U = x^{-1} \circ x\) on \(U\):}
                       & = \qty(f \circ x^{-1} \circ (x \circ \gamma))'(0)                          \\
                       & = \qty[\partial_i (f \circ x^{-1})(x(p))] \cdot \qty(x^i \circ \gamma)'(0) \\
                       & = (x^i \circ \gamma)'(0) \cdot \qty(\pdv{x^i})_p (f).
              \end{align*}
              Note that \(x^i \circ \gamma: \R \to \R\) is a smooth function of a real variable, so \((x^i \circ \gamma)'(0) \in \R\). Define
              \begin{equation}
                  X^i := (x^i \circ \gamma)'(0) \in \R, \quad i = 1, \ldots, d.
              \end{equation}
              So we have
              \begin{equation}
                  X(f) = X^i \qty(\pdv{x^i})_p (f) = X^i e_i(f), \quad \forall f \in \SC^\infty(M).
              \end{equation}
              Since this is true for all \(f \in \SC^\infty(M)\), we have
              \begin{equation}
                  X = X^i e_i.
              \end{equation}
              Thus, \(\ST_p M = \Span(\SB)\).

              \begin{remark}[Smoothness of Chart map and co-ordinate functions]
                  In general to talk about smooth of any function \(f: M \to \R\), we have used charts such that \(f\) is smooth if and only if \(f \circ x^{-1}: x(U) \subseteq \R^d \to \R\).

                  So by this definition, the chart map \(x: U \to x(U) \subseteq \R^d\) is trivially smooth, since \(x \circ x^{-1} = \id_{x(U)}\) is smooth. Similarly, the coordinate functions \(x^i: U \to \R\) are also smooth, since \(x^i \circ x^{-1}: x(U) \subseteq \R^d \to \R\) is just the projection onto the \(i\)-th coordinate, which is a linear map and hence smooth.
              \end{remark}

        \item \textbf{Linear Independence}: Suppose that \(\SB\) is linearly dependent, then there exist real numbers \(X^1, \ldots, X^d\), not all zero, such that
              \begin{equation}
                  X^j e_j = 0.
              \end{equation}
              So for all \(f \in \SC^\infty(M)\), we have \(X^j e_j(f) = 0\). In particular, take \(f = x^i\), the \(i\)-th coordinate function of chart \(x\), then
              \begin{align*}
                  0 & = X^j e_j(x^i) = X^j \qty(\pdv{x^j})_p (x^i) = X^j \partial_j (x^i \circ x^{-1})(x(p)) \\
                    & = X^j \partial_j (\proj^i) (x(p)) = X^j \delta_j^i = X^i.
              \end{align*}
              Since this is true for all \(i = 1, \ldots, d\), we have \(X^1 = X^2 = \ldots = X^d = 0\), which contradicts our assumption. Hence, \(\SB\) is linearly independent.
    \end{enumerate}
    Therefore, \(\SB\) is a basis of \(\ST_p M\), and \(\dim(\ST_p M) = d\).
\end{proof}
Terminology: Let \(X \in \ST_p M\), then we have
\begin{equation}
    X = X^j \qty(\pdv{x^j})_p,
\end{equation}
where \(X^j = X(x^j) = (x^j \circ \gamma)'(0)\) are called the \emph{components} of \(X\) with respect to the basis \(\SB\) induced by the chart \((U, x)\).

\begin{remark}[Dependence on Chart Choice]
    The basis \(\SB\) of \(\ST_p M\) constructed above depends on the choice of chart \((U, x)\). If we choose another chart \((V, y)\) around \(p\), then the transition map \(y \circ x^{-1}: x(U \cap V) \to y(U \cap V)\) is a diffeomorphism between open subsets of \(\R^d\). The new basis induced by chart \((V, y)\) will be
    \begin{equation}
        \tilde{\SB} = \qty{\qty(\pdv{y^1})_p, \ldots, \qty(\pdv{y^d})_p}.
    \end{equation}
    The relationship between the two bases can be expressed using the Jacobian matrix of the transition map. Specifically, for each \(j\),
    \begin{equation}
        \qty(\pdv{y^j})_p = \qty[\pdv{y^i}{x^j} \qty(x(p))] \qty(\pdv{x^i})_p,
    \end{equation}
    where \(\pdv{y^i}{x^j}(x(p))\) is the \((i,j)\)-th entry of the Jacobian matrix of the transition map \(y \circ x^{-1}\) evaluated at \(x(p)\). This shows how the basis of the tangent space transforms under a change of coordinates.
\end{remark}
Usually in physics, when we talk about position vector, we mean the coordinates functions \(x^i\) in some chart. Say we go to a different chart \(y\) (coordinate transformation) which has high non-linear dependence on \(x\), then the position vector in the new chart \(y\) is not simply related to the old position vector in chart \(x\) by a linear transformation, but rather by a non-linear transformation. However, the tangent vectors (velocity vectors) transform linearly under the change of coordinates, as shown above. So the notion of position vector, and its transformation is ill-defined in general.

This becomes very important when we study general relativity, where the spacetime is modeled as a 4-dimensional Lorentzian manifold. In some older physics literature, you may find the term position vector being used for the coordinates \(x^\mu\), which is not a well-defined concept.



\end{document}
