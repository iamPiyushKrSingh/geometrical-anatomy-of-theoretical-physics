\lecture{2}

\section{The \texorpdfstring{\(\in\)}{in}-Relation}

Set theory built on the postulate that there is a fundamental relation\footnote{A predicate of two variables} called \(\in\). There will be no definition of what \(\in\) is, or what a set is.

\noindent Instead of defining what a set is, we have 9 axioms that speak of \(\in\) and sets. Overviews of these axioms are as follows:
\begin{table}[H]
	\centering
	\def\arraystretch{1.5}\tabcolsep=10pt
	\begin{tabular}{|c|c|}
		\hline
		E & \multirow{2}{*}{basic existence axiom}              \\
		E &                                                     \\
		\hline
		P & \multirow{4}{*}{Construction axiom}                 \\
		U &                                                     \\
		R &                                                     \\
		P &                                                     \\
		\hline
		I & \multirow{2}{*}{further existence and construction} \\
		C &                                                     \\
		\hline
		F & Non-existence axiom                                 \\
		\hline
	\end{tabular}
	\caption{Mnemonic for Axioms of Set Theory}
\end{table}\noindent
Using the \(\in\)-relation, we can immediately define following relations:
\begin{flalign}
	 & \textsf{\textbf{Not an element}:} \quad x \notin y :\iff \neg(x \in y)                                              \label{eq:notin} &  & \\
	 & \textsf{\textbf{Subset}:}         \quad x \subseteq y :\iff \forall z: (z \in x \implies z \in y) \label{eq:subset}                  &  & \\
	 & \textsf{\textbf{Equality}:}       \quad x = y :\iff x \subseteq y \land y \subseteq x                              \label{eq:equal}  &  &
\end{flalign}

\section{Zermelo-Fraenkel Axioms of Set Theory}
\begin{axiom}[Axiom on \(\in\)-relation]\label{axiom:E1}
	\(x \in y\) is a proposition if and only if \(x\) and \(y\) are both sets.
	\begin{equation}
		\forall x: \forall y: (x \in y) \lxor \lnot (x \in y)
	\end{equation}
\end{axiom}\noindent
\textsf{\textbf{Counter example} (Russell's Paradox):} Assume there is some \(u\) that contains all sets that do not contain themselves as an element. To be precise,
\begin{equation}
	\exists u: \forall z: (z \in u \iff z \notin z)
\end{equation}
Problem: Is \(u\) a set?\\
If \(u\) was a set, then one must be able to determine whether \(u \in u\) is true or false. As \cref{axiom:E1} states, \(u \in u\) is a proposition if and only if \(u\) is a set.
\begin{description}
	\item[Case 1:] Assume \(u \in u\) is true: \(\xRightarrow[\stackrel{\text{def of}}{u}]{} u \notin u\). \(\contra\)

	\item[Case 2:] Assume \(u \in u\) is false: \(\xLeftrightarrow[\stackrel{\text{def}}{\notin}]{} u \notin u \xRightarrow[\stackrel{\text{def}}{u}]{} u \in u\). \(\contra\)
\end{description}
Conclusion: \(u\) is not a set. \hfill \(\square\)

\begin{axiom}[Existence of an empty set]\label{axiom:E2}
	There exists a set that contains no elements.
	\begin{equation}
		\exists x: \forall y: y \notin x
	\end{equation}
\end{axiom}

\begin{theorem}[Uniqueness of the empty set]\label{thm:empty_set}
	There is only one empty set. And it is denoted by \(\0\).
\end{theorem}
\begin{proof}[Standard textbook style]
	Assume \(x\) and \(x'\) are both empty sets. But then
	\begin{equation*}
		\forall y: \overbrace{\underbrace{(y \notin x)}_{\mathclap{\text{always false}}} \implies (y \in x')}^{\mathclap{\text{since hypothesis is always false the implication is always true}}}
	\end{equation*}
	But this just means that \(x \subseteq x'\) from \cref{eq:subset}. Conversely,
	\begin{equation*}
		\forall y: (y \notin x') \implies (y \in x)
	\end{equation*}
	But thus \(x' \subseteq x\). Hence, \(x = x'\).
\end{proof}
The above proof is a bit informal. Now we will prove it using the definition of proof in an axiomatic system.
\begin{proof}[Formal Version]
	Suffices to show that if \(x\) and \(x'\) are both empty sets, then \(x = x'\).
	\begin{flalign*}
		 & \left.\mqty{ a_1 \iff \forall y: y \notin x                                                                    \\ a_2 \iff \forall y: y \notin x'}\right\} \text{assumptions or axioms} &                                                                                 &     \\
		 & q_1 \xLeftrightarrow[\text{(T)}]{} \forall y: y \notin x \implies \forall y: (y \in x \implies y \in x')  &  & \\
		 & q_2 \xLeftrightarrow[\text{(A)1}]{} \forall y: y \notin x                                                 &  & \\
		 & q_3 \xLeftrightarrow[\text{(M)1,2}]{} \forall y: (y \in x \implies y \in x')\ \boxed{\iff x \subseteq x'} &  & \\
		 & q_4 \xLeftrightarrow[\text{(T)}]{} \forall y: y \notin x' \implies \forall y: (y \in x' \implies y \in x) &  & \\
		 & q_5 \xLeftrightarrow[\text{(A)2}]{} \forall y: y \notin x'                                                &  & \\
		 & q_6 \xLeftrightarrow[\text{(M)4,5}]{} \forall y: (y \in x' \implies y \in x)\ \boxed{\iff x' \subseteq x} &  & \\
		 & q_7 \xLeftrightarrow[\text{(M)3,6}]{} x = x'                                                              &  &
	\end{flalign*}
\end{proof}

\begin{axiom}[Axiom on Pair Sets]\label{axiom:P3}
	Let \(x\) and \(y\) be sets. Then there exists a set that contains as its elements precisely the sets \(x\) and \(y\).
	\begin{equation}
		\forall x: \forall y: \exists m: \forall u: (u \in m \iff u = x \lor u = y)
	\end{equation}
	\uline{Notation:} denote this set \(m\) by \(\qty{x, y}\).
\end{axiom}

\begin{theorem}[Ordering in Pair Sets]\label{thm:order_pair}
	Let \(x\) and \(y\) be sets. Then \(\qty{x, y} = \qty{y, x}\).
\end{theorem}
\begin{proof}
	\begin{align*}
		a \in \qty{x, y} & \implies a \in \qty{y, x} \implies \qty{x, y} \subseteq \qty{y, x} \\
		a \in \qty{y, x} & \implies a \in \qty{x, y} \implies \qty{y, x} \subseteq \qty{x, y}
	\end{align*}
	Thus, \(\qty{x, y} = \qty{y, x}\).
\end{proof}

\begin{definition}[Set with one element]
	Let \(x\) be a set. Then there exists a set that contains as its element precisely the set \(x\).
	\begin{equation}
		\qty{x} := \qty{x, x}
	\end{equation}
\end{definition}

\begin{axiom}[Axiom on Union Sets]\label{axiom:U4}
	Let \(x\) be a set. Then there exists a set whose elements are precisely the elements of the elements of \(x\).
	\begin{equation}
		\forall x: \exists u: \forall y: (y \in u \iff \exists z: (z \in x \land y \in z))
	\end{equation}
	\uline{Notation:} denote this set \(u\) by \(\bigcup x\).
\end{axiom}

\begin{example}
	Let \(a, b\) be sets. Then by \cref{axiom:P3}, \(\qty{a}\) and \(\qty{b}\) are sets, and hence \(x := \qty{\qty{a}, \qty{b}}\) is a set. Then \(\bigcup x = \qty{a, b}\).
\end{example}
Observe that, since \(a\) and \(b\) are sets, then by \cref{axiom:P3}, \(\qty{a, b}\) is a set. So it gives us a false impression that \cref{axiom:U4} is redundant. But it is not. Consider the following example:
\begin{example}
	Let \(a, b, c\) be sets. Then by \cref{axiom:P3}, \(\qty{a, b}\) and \(\qty{c}\) are sets, and hence \(x := \qty{\qty{a, b}, \qty{c}}\) is a set. Then
	\begin{equation}
		\bigcup x =: \qty{a, b, c}.
	\end{equation}
\end{example}
With this example, we can generalize the definition of a finite set.

\begin{definition}[Finite Set]
	Let \(a_1, a_2, \ldots, a_N\) be sets. Define, recursively for all \(N \ge 3\),
	\begin{equation}
		\qty{a_1, a_2, \ldots, a_N} := \bigcup\qty{\qty{a_1, a_2, \ldots, a_{N-1}}, \qty{a_N}}
	\end{equation}
\end{definition}

\begin{axiom}[Axiom of Replacement]\label{axiom:R5}
	Let \(R\) be a functional relation. Let \(m\) be a set. Then the image of \(m\) under \(R\) \normalfont{[\(\im_R(m)\)]} is a set.
\end{axiom}
For this axiom to make sense, we need to define what a functional relation is.
\begin{definition}[Functional Relation]
	A relation \(R\) is functional if and only if
	\begin{equation}
		\forall x: \exists!\ y: R(x, y)\footnotemark
	\end{equation}
\end{definition}
\footnotetext{The symbol \(\exists!\) is `Uniqueness' quantifier and read as ``there exists a unique'' and is a shorthand for \(\exists y: R(x, y) \land \forall y': (R(x, y') \implies y = y')\).}
We define the image of a set under a functional relation keeping in mind that \cref{axiom:R5} guarantees this object to be a set.
\begin{definition}[Image of a Set under a Functional Relation]
	Let \(R\) be a functional relation. Let \(m\) be a set. Then the image of \(m\) under \(R\) consists of all those elements \(y\) for which there is an element \(x \in m\) such that \(R(x, y)\).
\end{definition}

The axiom of replacement is a very powerful axiom. It implies, but is not implied by, the ``\emph{principle of restricted comprehension}''.
\begin{theorem}[Principle of Restricted Comprehension (PRC)]\label{thm:PRC}
	Let \(P\) be a predicate of one variable and let \(m\) be a set. Then those elements \(y \in m\) for which \(P(y)\) is true constitute a set.

	\noindent\uline{Notation:} this set is denoted by \(\qty{y \in m \mid P(y)}\).
\end{theorem}
PRC is not to be confused with the inconsistent ``\emph{principle of unrestricted comprehension}'' (PUC) which allows for paradoxes like `Self-Reference Paradox.'

Consider the predicate \(P(x)\) is true if \(x\) is a set that doesn't contain themselves as an element. Thus, from PUC we have \(\qty{y \mid P(y)}\) is a set (sick), but this give rise to Russell's Paradox in `\emph{naive set theory}.'

Intuitively PRC makes sure that the image set doesn't grow bigger than a set itself.
\begin{proof}
	We can prove PRC using the axiom of replacement in following cases:
	\begin{description}
		\item[Case 1:] \(\lnot \exists y \in m: P(y)\)\footnote{The quantifier \(\exists y \in m: P(y)\) is logically equivalent to \(\lnot(\forall y \in m: \lnot P(y))\).} in this case,
		      \begin{equation*}
			      \qty{y \in m \mid P(y)} := \0
		      \end{equation*}

		\item[Case 2:] \(\exists \hat{y} \in m: P(\hat{y})\), then define a relation \(R\) as follows:
		      \begin{equation*}
			      R(x, y) := \qty(P(x) \land x = y) \lor \qty(\lnot P(x) \land \hat{y} = y)
		      \end{equation*}
		      \begin{claim}
			      \(R\) is a functional relation.
		      \end{claim}
		      \begin{Proof}
			      Let \(x\) be a set. If \(P(x)\) is true, then \(R(x, y)\) is true if and only if \(y = x\). If \(P(x)\) is false, then \(R(x, y)\) is true if and only if \(y = \hat{y}\). \\
			      Thus, \(\forall x: \exists! y: R(x, y)\). Hence, \(R\) is a functional relation.
		      \end{Proof}
		      Then by \cref{axiom:R5}, \(\im_R(m)\) is a set. And by definition of \(R\), \(\forall y \in \im_R(m): P(y)\)\footnotemark, thus we can define \(\qty{y \in m \mid P(y)} := \im_R(m)\).
	\end{description}
\end{proof}
\footnotetext{The quantifier \(\forall y \in m: P(y)\) is logically equivalent to \(\forall y: (y \in m \implies P(y))\).}

\begin{definition}[Set Difference]
	Let \(u\) and \(m\) be sets such that \(u \subseteq m\). Then the set difference of \(m\) and \(u\) is defined as
	\begin{equation}
		m \setminus u := \qty{x \in m \mid x \notin u}.
	\end{equation}
\end{definition}
The object \(m \setminus u\) is a set due to PRC \ie\ ultimately due to the axiom of replacement.

\begin{definition}[Intersection of Sets]
	Let \(x\) be a set. Then the intersection of \(x\) is defined as
	\begin{equation}
		\bigcap x := \qty{y \in \bigcup x \mid \forall z \in x: y \in z}
	\end{equation}
\end{definition}
Historically, in naive set theory, PUC was thought to be needed in order to define, for any set \(m\),
\begin{equation*}
	\powerset(m) \underset{\text{inconsistently}}{:=} \qty{u \mid u \subseteq m}
\end{equation*}
This definition is circular, as we need to know a priori, from which bigger set the elements of \(\powerset(m)\) are taken.

\begin{axiom}[Axiom on existence of power sets]\label{axiom:P6}
	Let \(m\) be a set. Then there exists a set, denoted by \(\powerset(m)\), whose elements are precisely the subsets of \(m\).
	\begin{equation}
		\forall m: \exists p: \forall u: (u \in p \iff u \subseteq m)
	\end{equation}
\end{axiom}
\begin{example}
	Let \(m = \qty{a, b}\), then \(\powerset(m) = \qty{\0, \qty{a}, \qty{b}, \qty{a, b}}\).
\end{example}
At this point, we can define what an ordered pair is and what a Cartesian product is.
\begin{definition}[Ordered Pair]
	Let \(X\) and \(Y\) be non-empty sets. For any \(x \in X\) and \(y \in Y\), define the ordered pair
	\begin{equation}
		(x, y) := \qty{\qty{x}, \qty{x, y}}
	\end{equation}
\end{definition}
From this definition, we can prove the following theorem which justifies the name `ordered pair.'
\begin{theorem}[Ordering in Ordered Pairs]
	Let \(X\) and \(Y\) be non-empty sets. Then \((x, y) = (x', y')\) if and only if \(x = x'\) and \(y = y'\).
\end{theorem}
\begin{proof}

	[\(\implies\)]\\
	We will prove this theorem in two cases:\\
	\textsf{\textbf{Case 1:}} Assume \(x = y\):
	\begin{gather*}
		(x, y) = \qty{\qty{x}, \qty{x, y}} = \qty{\qty{x}, \qty{x, x}} = \qty{\qty{x}, \qty{x}} = \qty{\qty{x}} \\
		\qty{x', y'} \in \qty{\qty{x'}, \qty{x', y'}} = (x', y') = (x, y) = \qty{\qty{x}} \implies \qty{x', y'} = \qty{x}
	\end{gather*}
	Thus by \cref{axiom:P3}, \(x' = y' = x\).\\
	\textsf{\textbf{Case 2:}} Assume \(x \neq y\):\\
	Observe that \(x' \ne y'\) if not then \(\qty{x, y} \in (x, y) = (x', y') = \qty{\qty{x'}} \implies x = y = x'\). \(\contra\) \\
	As \(\qty{x'} \in (x', y') = (x, y)  = \qty{\qty{x}, \qty{x, y}}\), we have \(\qty{x'} = \qty{x}\) as \(\qty{x'} \ne \qty{x, y}\) since \(x \ne y\). Hence, \(x' = x\). \\
	Now, \(\qty{x', y'} \in (x', y') = (x, y) = \qty{\qty{x}, \qty{x, y}}\), then \(\qty{x', y'} = \qty{x}\) or \(\qty{x', y'} = \qty{x, y}\). Observe \(\qty{x', y'} \ne \qty{x}\) if not then \(x' = y' = x\). \(\contra\) \\
	Hence \(\qty{x', y'} = \qty{x, y}\) and thus \(y' = y\) as \(y' \ne x\) as \(x' \ne y'\). Thus, \(x' = x\) and \(y' = y\).

	\noindent[\(\impliedby\)]\\
	Trivial.
\end{proof}
Is collection of all ordered pairs a set? Answer to this question leads us to the definition of Cartesian product.
\begin{theorem}[Existence of Cartesian Product]
	Let \(X\) and \(Y\) be non-empty sets. Then the collection of all ordered pairs of elements of \(X\) and \(Y\) is a set \ie
	\begin{equation}
		X \times Y := \qty{(x, y) \mid x \in X \land y \in Y}
	\end{equation}
	is a set.
\end{theorem}
\begin{proof}
	Suffice to show that for any \(x \in X\) and \(y \in Y\), \((x, y)\) exists in ``bigger'' set that we know exists. Then by PRC, \(X \times Y\) is a set.

	\noindent Fix \(x \in X\) and \(y \in Y\). Then by \cref{axiom:P3}, \(\qty{x}\) and \(\qty{y}\) are sets. Then by \cref{axiom:P3}, \(\qty{x, y}\) is a set. So by \cref{eq:subset}, \(\qty{x} \subseteq X \implies \qty{x} \subseteq \bigcup \qty{X, Y}\) and \(\qty{x, y} \subseteq \bigcup \qty{X, Y}\). So by \cref{axiom:P6} we know \(\powerset(\bigcup\qty{X, Y})\) exists. Thus,
	\begin{equation}
		\qty{x} \in \powerset\qty(\bigcup\qty{X, Y}) \quad \text{and} \quad \qty{x, y} \in \powerset\qty(\bigcup\qty{X, Y})
	\end{equation}
	With this we can write \((x, y) = \qty{\qty{x}, \qty{x, y}} \subseteq \powerset\qty(\bigcup\qty{X, Y})\). Hence, \((x, y) \in \powerset\qty(\powerset\qty(\bigcup\qty{X, Y}))\). Thus, by PRC,
	\begin{equation}
		X \times Y = \qty{u \in \powerset\qty(\powerset\qty(\bigcup\qty{X, Y})) \mid \exists x \in X: \exists y \in Y: u = (x, y)}
	\end{equation}
	is a set.
\end{proof}

\begin{axiom}[Axiom of Infinity]\label{axiom:I7}
	There exists a set that contains the empty set as an element and with every of its elements \(y\) it also contains the set \(\qty{y}\) as an element.
	\begin{equation}
		\exists x: \0 \in x \land \forall y: (y \in x \implies \qty{y} \in x).
	\end{equation}
\end{axiom}

\begin{remark}[Smallest Infinite Set]
	Let \(x\) be the set guaranteed by \cref{axiom:I7}. Then \(\0 \in x \implies \qty{\0} \in x \implies \cdots\),
	\begin{equation}
		x = \qty{\0, \qty{\0}, \qty{\qty{\0}}, \qty{\qty{\qty{\0}}} \ldots}
	\end{equation}
\end{remark}
Notationally, writing down the set \(x\) is cumbersome as there are too many braces. So we denote each element of \(x\) by a natural number as follows:
\begin{equation}
	0 := \0, \quad 1 := \qty{\0}, \quad 2 := \qty{\qty{\0}}, \quad 3 := \qty{\qty{\qty{\0}}}, \quad \ldots
\end{equation}
Thus, the set \(x\) can be written as
\begin{equation}
	x = \qty{0, 1, 2, 3, \ldots}
\end{equation}
\begin{corollary}[Existence of Natural Numbers]
	\(\N\) is a set.
\end{corollary}

\begin{remark}[Set of Real Numbers]
	As a set, \(\R = \powerset(\N)\).
\end{remark}

\begin{axiom}[Axiom of Choice]\label{axiom:C8}
	Let \(x\) be a set whose elements are non-empty and mutually disjoint sets, then there exists a set \(y\) which contains precisely one element from each element of \(x\).
	\begin{equation}
		\forall x: P(x) \implies \exists y: \forall w \in x: \exists! z \in w: z \in y
	\end{equation}
	where, \(P(x) :\iff (\exists u: u \in x) \land \qty(\forall u \in x: \forall v \in x: u \neq v \implies \bigcap \qty{u, v} = \0)\).
\end{axiom}
The set \(y\) is called ``\emph{dark}'' set. Say we have a set \(x\) that contains pair of shoes, then we can algorithmically choose the left shoe from each pair of shoes to form the set \(y\). But if we have a set \(x\) that contains a pair of socks, then we can't algorithmically choose one sock from each pair of socks to form the set \(y\). This is where the axiom of choice comes into play.

The axiom of choice is independent of the other 8 axioms, which means that one could have set theory with or without the axiom of choice. However, standard mathematics uses the axiom of choice and hence so will we.

There is a number of theorems that can only be proved by using the axiom of choice. Amongst these we have:
\begin{itemize}
	\item Proof that every vector space has a basis needs the axiom of choice.
	\item Proof that there exists a complete system of representatives for the equivalence classes of a set under an equivalence relation needs the axiom of choice.
\end{itemize}

\begin{axiom}[Axiom of foundation]\label{axiom:F9}
	Every non-empty set \(x\) contains an element \(y\) that has none of its elements in common with \(x\).
	\begin{equation}
		\forall x: (\exists y: y \in x) \implies \exists y \in x: \bigcap \qty{x, y} = \0
	\end{equation}
\end{axiom}

\begin{corollary}
	There is no set that contains itself as an element.
	\begin{equation}
		x \in x \quad \text{is false for all sets \(x\)}
	\end{equation}
\end{corollary}
The totality of these 9 axioms is called ZFC (Zermelo-Fraenkel with the axiom of choice) set theory.
