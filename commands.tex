%% New Commands %%
\usetikzlibrary{arrows.meta}
\usepackage{xparse}

\NewDocumentCommand{\lecture}{om}{%

	\makeatletter
	\addtocontents{toc}{\def\string\@dotsep{1000}}
	\makeatother

	\noindent\rule[0.35em]{0.4\textwidth}{1pt} \hfill \fbox{\large\textsc{Lecture #2}} \hfill \rule[0.35em]{0.4\textwidth}{1pt}
	\IfNoValueTF{#1}
	{
		\addcontentsline{toc}{section}{%
			% \protect\newline{}%
			\sffamily{\bfseries Lecture #2}
		}
	}
	{
		\addcontentsline{toc}{section}{%
			% \protect\newline{}%
			\sffamily{\bfseries Lecture #2:} #1
		}
	}

	\makeatletter
	\addtocontents{toc}{\def\string\@dotsep{4.5}}  % 4.5 might be a different value in your class
	\makeatother
}

% Ch01 - Axiomatic Set Theory %
\DeclareMathOperator{\id}{id}
\DeclareMathOperator{\lxor}{\veebar}
\DeclareMathOperator{\lnand}{\uparrow}
\usepackage{stmaryrd}
\DeclareMathOperator{\contra}{\lightning}
% \DeclareMathOperator{\im}{im} % already defined as \img
\newcommand{\0}{\emptyset}
\DeclareMathOperator{\preimg}{preim}
\DeclareMathAlphabet{\euscr}{U}{eus}{m}{n}
\DeclareMathOperator{\powerset}{\euscr{P}}
\DeclareMathOperator{\setIso}{\cong_{\text{set}}}
\usetikzlibrary{cd}
\renewcommand{\nsim}{\not\sim}
\newcommand{\into}{\hookrightarrow}

% Ch02 - Topological Spaces %
\renewcommand{\O}{\mathcal{O}}
\newcommand{\Ostd}{\mathcal{O}_{\text{std.}}}
\newcommand{\Oq}[1]{\mathcal{O}_{\faktor{#1}{\sim}}}
\DeclareMathOperator{\topIso}{\cong_{\text{top.}}}
\DeclareMathOperator{\supp}{supp}
\DeclareMathOperator{\dcup}{\dot{\cup}}
\DeclareMathOperator{\grpIso}{\cong_{\text{grp}}}

% Ch03 - Topological Manifolds and Bundles %
\usetikzlibrary{decorations.markings}
\DeclareMathOperator{\bdlIso}{\cong_{\text{bdl}}}
\DeclareMathOperator{\proj}{proj}
\newcommand{\A}{\mathscr{A}}
\newcommand{\I}{\mathrm{i}}

% Ch04 - Differentiable Manifolds %
\usepackage{bbding}
\newcommand{\flower}{{\scalebox{0.75}{\FiveFlowerOpen} }}
\DeclareMathOperator{\diffIso}{\cong_{\text{diff}}}
\DeclareMathOperator{\vecIso}{\cong_{\text{vec}}}
\DeclareMathOperator{\Hom}{Hom}
\DeclareSymbolFont{MnSyC}{U}{MnSymbolC}{m}{n}
\DeclareMathSymbol{\diamondplus}{\mathbin}{MnSyC}{"7C}
\DeclareMathSymbol{\diamonddot}{\mathbin}{MnSyC}{"7E}
\DeclareMathOperator{\End}{End}
\DeclareMathOperator{\Aut}{Aut}
\usepackage{tensor}
\DeclareMathOperator{\hateq}{\hat{=}}
\DeclareMathOperator{\sgn}{sgn}
\DeclareMathOperator{\vol}{vol}

\DeclareFontFamily{U}{MnSymbolC}{}
\DeclareFontShape{U}{MnSymbolC}{m}{n}{
	<-6>  MnSymbolC5
	<6-7>  MnSymbolC6
	<7-8>  MnSymbolC7
	<8-9>  MnSymbolC8
	<9-10> MnSymbolC9
	<10-12> MnSymbolC10
	<12->   MnSymbolC12}{}
