\lecture[Construction of Tangent Bundle]{10}

\section{Cotangent Spaces and Gradient}

\begin{definition}[Cotangent Space]
	Let \(M\) be a smooth manifold and \(p \in M\). The \emph{cotangent space} at \(p\), denoted \(\ST_p^* M\), is the dual space of the tangent space \(\ST_p M\):
	\begin{equation}
		\ST_p^* M := (\ST_p M)^*.
	\end{equation}
\end{definition}
Since the manifold is finite-dimensional\footnotemark, the tangent space \(\ST_p M\) is a finite-dimensional vector space. Therefore, its dual space \(\ST_p^* M\) is also a finite-dimensional vector space of the same dimension, moreover, we have the isomorphism
\begin{equation}
	\ST_p M \vecIso \ST_p^* M.
\end{equation}
\footnotetext{The way we have defined manifolds in this course, they are finite-dimensional by default as we always map them to \(\R^d\). For infinite-dimensional manifolds, we need to more sophisticated tools like Banach and Hilbert spaces, which are beyond the scope of this course.}%
However, this isomorphism is not canonical, \ie, there is no natural way to identify vectors in \(\ST_p M\) with covectors in \(\ST_p^* M\) without introducing some additional structure on these tangent and cotangent spaces.

Now, using these vector and dual vector spaces, we can define tensor spaces at a point \(p \in M\),
\begin{equation}
	\mathsf{T}^r_s(\ST_p M) := \qty{t : \underbrace{\ST_p^* M \times \dots \times \ST_p^* M}_{r\text{ times}} \times \underbrace{\ST_p M \times \dots \times \ST_p M}_{s\text{ times}} \to \R \mid t \text{ is multilinear}}.
\end{equation}
This is set (even vector space) of all \((r,s)\)-type tensors at the point \(p\).

\begin{definition}[Gradient]
	Let \(f \in \SC^\infty(M)\). Then at each point \(p \in M\), we have a linear map
	\begin{equation}
		\begin{aligned}
			\dd_p : \SC^\infty(M) & \to \ST_p^* M    \\
			f                     & \mapsto \dd_p f,
		\end{aligned}
	\end{equation}
	where \(\dd_p f\) is defined by
	\begin{equation}
		\dd_p f (X) := X(f), \quad \forall X \in \ST_p M.
	\end{equation}
	The linear map \(\dd_p\) is called the \emph{gradient operator} at the point \(p\), and \(\dd_p f\) is called the \emph{gradient} of the function \(f\) at the point \(p\).
\end{definition}
Observe that the gradient \(\dd_p f\) is a covector at the point \(p\) not a vector. Also, the gradient of a function \(f\) at a point \(p\) is ``\emph{orthogonal}'' to the level set of \(f\) passing through the point \(p\), \ie, let a tangent vector \(X \in \ST_p M\) be tangent to the level set\footnotemark\ of \(f\) at the point \(p\), then we have
\begin{equation*}
	\dd_p f (X) = X(f) = 0,
\end{equation*}
since the function \(f\) is constant on its level set. Thus, the gradient \(\dd_p f\) annihilates all tangent vectors to the level set of \(f\) at the point \(p\).
\footnotetext{The level set of a function \(f : M \to \R\) corresponding to a value \(c \in \R\) is defined as the set \(\mathfrak{L}_c(f) := \qty{p \in M \mid f(p) = c}\).}%

\subsection{Basis of Cotangent Space}

We have defined the gradient operator at a point \(p \in M\), using which we can construct a basis for the cotangent space \(\ST_p^* M\).
\begin{theorem}[Basis of Cotangent Space]
	Let \(M\) be a smooth manifold and \(p \in M\). Let \((U, x)\) be a chart around the point \(p\). Then the set
	\begin{equation}
		\qty{\dd_p x^1, \dd_p x^2, \dots, \dd_p x^{\dim M}}
	\end{equation}
	where \(x^i\) is the \(i\)-th component function of the chart \(x\), forms a basis of the cotangent space \(\ST_p^* M\).
\end{theorem}
\begin{proof}
	Since the dimension of the cotangent space \(\ST_p^* M\) is equal to the dimension of the manifold \(M\), it is sufficient to show that the set \(\qty{\dd_p x^i}_{i = 1}^{\dim M}\) is linearly independent. Suppose not the set is linearly dependent, then there exist real numbers \(\omega_i \in \R\), not all zero, such that
	\begin{equation*}
		\omega_i \dd_p x^i = 0. \qquad (\text{sum over } i \text{ from } 1 \text{ to } \dim M \text{ is implied}).
	\end{equation*}
	So for any tangent vector \(X \in \ST_p M\), we have \(\omega_i \dd_p x^i (X) = 0\). In particular, let us choose \(X = \qty(\pdv{x^j})_p\), the \(j\)-th basis vector of the tangent space \(\ST_p M\) induced by the chart \((U, x)\). Then we have
	\begin{align*}
		0 & = \omega_i \dd_p x^i \qty(\qty(\pdv{x^j})_p) = \omega_i \qty(\pdv{x^j})_p (x^i) = \omega_i \partial_j (x^i \circ x\inv) (x(p)) \\
		  & = \omega_i \partial_j (\proj^i) (x(p)) = \omega_i \delta^i_j = \omega_j.
	\end{align*}
	Here, \(\proj^i : \R^{\dim M} \to \R\) is the projection map onto the \(i\)-th coordinate. Since \(j\) was arbitrary, we have \(\omega_j = 0\) for all \(j = 1, 2, \dots, \dim M\), which contradicts our assumption that not all \(\omega_i\) are zero. Thus, the set \(\qty{\dd_p x^i}_{i = 1}^{\dim M}\) is linearly independent and hence forms a basis of the cotangent space \(\ST_p^* M\).
\end{proof}
Moreover, we have the following duality relation between the basis of the tangent space and the basis of the cotangent space induced by the same chart \((U, x)\):
\begin{equation}
	\dd_p x^i \qty(\qty(\pdv{x^j})_p) = \delta^i_j.
\end{equation}
Thus, the basis \(\qty{\dd_p x^i}_{i = 1}^{\dim M}\) is \emph{dual basis} to the basis \(\qty{\qty(\pdv{x^i})_p}_{i = 1}^{\dim M}\).

\section{Push-Forward and Pull-Back}
