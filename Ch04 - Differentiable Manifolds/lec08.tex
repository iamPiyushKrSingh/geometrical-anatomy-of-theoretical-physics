\lecture{8}

\section{Review of Vector Spaces}

\begin{definition}[(Algebraic) Field]
    Let \(K\) be a non-empty set with two binary operations \(+\) and \(\cdot\) (addition and multiplication) such that
    \begin{itemize}
        \item \((K, +)\) is an abelian group with identity element \(0\).
        \item \((K \setminus \{0\}, \cdot)\) is an abelian group with identity element \(1\).
        \item Multiplication is distributive over addition, \ie\ for all \(a, b, c \in K\),
              \begin{equation}
                  a \cdot (b + c) = a \cdot b + a \cdot c \quad \text{and} \quad (a + b) \cdot c = a \cdot c + b \cdot c.
              \end{equation}
    \end{itemize}
    Then \(K\) is called a field.
\end{definition}
Later we will use a set \(R\) equipped with two binary operations \(+\) and \(\cdot\) but fewer axioms than a field, and we will call it a \emph{ring}.
\begin{definition}[Ring]
    A ring is a set \(R\) equipped with two binary operations \(+\) and \(\cdot\) such that
    \begin{itemize}
        \item \((R, +)\) is an abelian group with identity element \(0\).
        \item \((R, \cdot)\) is a monoid with identity element \(1\).

              \ie\ multiplication is associative and has an identity element, but it need not be commutative.
        \item Multiplication is distributive over addition, \ie\ for all \(a, b, c \in R\),
              \begin{equation}
                  a \cdot (b + c) = a \cdot b + a \cdot c \quad \text{and} \quad (a + b) \cdot c = a \cdot c + b \cdot c.
              \end{equation}
    \end{itemize}
\end{definition}
Some examples which we are using from our school days:
\begin{itemize}
    \item \((\Z, +, \cdot)\) is a \emph{commutative} ring but not a field as it has no multiplicative inverses for all non-zero elements.
    \item \((\Q, +, \cdot)\) and \((\R, +, \cdot)\) are fields.
    \item \((\SM_{n}(\R), +, \circ)\) is a ring, where \(\SM_{n}(\R)\) is the set of \(n \times n\) real matrices with the usual matrix addition and multiplication. Matrix multiplication is associative but not commutative, and it has an identity element \(I_n\) (the identity matrix). But it does not have multiplicative inverses for all non-zero elements, so it is not a field.
\end{itemize}

\begin{definition}[Vector Space over a Field \(K\)]
    A vector space \(V\) over a field \((K, +, \cdot)\) is a set equipped with two operations \(\oplus: V \times V \to V\) (vector addition) and \(\odot: K \times V \to V\) (scalar multiplication) such that
    \begin{itemize}
        \item \((V, \oplus)\) is an abelian group with identity element \(\va{0}\) (the zero vector).
        \item The scalar multiplication satisfies the following properties
              \begin{enumerate}[(i)]
                  \item Let \(1\) be the multiplicative identity of the field \(K\). Then for all \(\va{v} \in V: 1 \odot \va{v} = \va{v}\).
                  \item \(\forall a, b \in K: \forall \va{v} \in V: a \odot (b \odot \va{v}) = (a \cdot b) \odot \va{v}\).
                  \item \(\forall a \in K: \forall \va{u}, \va{v} \in V: a \odot (\va{u} \oplus \va{v}) = a \odot \va{u} \oplus a \odot \va{v}\).
                  \item \(\forall a, b \in K: \forall \va{v} \in V: (a + b) \odot \va{v} = a \odot \va{v} \oplus b \odot \va{v}\).
              \end{enumerate}
    \end{itemize}
\end{definition}
